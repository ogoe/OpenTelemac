\section{Sand grading effects}\label{ch:sandgrading}
\subsection{Sediment bed composition}

\subsubsection{Granulometry distribution}
The number of classes of bed material is specified in the steering file and
limited to \texttt{NSICLA} $\leq 20$. For uniform sediment, the granulometry distribution is represented by one
single value \texttt{NSICLA} $= 1$ for the whole domain. The mean grain diameter and
corresponding settling velocity are defined in the steering file. 

For sediment mixtures, the granulometry distribution is discretized in a
number of classes. Each class of sediment \texttt{1 < j < NSICLA} is defined by its mean diameter $d_{50}$\texttt{(j)} and volume fraction in the mixture at every node \texttt{AVAI(j)}. The characteristics of each class
of sediment, for example the Shields number $\theta_c$\texttt{(i)}, settling
velocity $w_s$\texttt{(i)} of each class can be specified in the steering file
or calculated by the model, as for a single sediment class. 

The percent of each class of material need to verify, such that:
\begin{equation}\label{eq:percentclass}
\sum_{\mathtt{j=1,NSICLA}} \mathtt{AVAI(j)} = 1 \quad \text{and} \quad 0\leq \mathtt{AVAI(j)} \leq 1. 
\end{equation}

The initial bed composition can be defined in the steering file for uniform bed. 
For complex bed configuration, e.g. spatial
variation, vertical structure, etc., the user subroutine \texttt{init\_compo.f} is used to define the initial state.

The mean diameter $D_m$, \texttt{ACLADM}, is calculated as:
\begin{equation}\label{eq:d50class}
D_m = \sum_{\mathtt{j=1,NSICLA}} \mathtt{AVAI(j)\,D(j)}. 
\end{equation}

\subsubsection{Bed structure}
The bed is stratified into a number of layers \texttt{NOMBLAY} that enables
vertical variations of the sediment bed composition. The percentage of each
class \texttt{j} of material, \texttt{AVAI(i,j,k)} as well as the thickness of each
layer $E_s$ are defined at each point \texttt{i} and for each layer \texttt{k}. 

The composition of transported material is computed using the composition of
the upper layer, see below the definition of \emph{active layer}. The initial composition of the bed (number of layers, thickness of each layer $E_s$, composition of each layer \texttt{AVAI} are specified in user's subroutine \texttt{init\_compo.f}.

The subroutine \texttt{init\_avail.f} verifies if the structure of the initial bed
composition is compatible with the position of the rigid bed, as defined in
user's subroutine \texttt{noerod.f}):

\begin{equation}\label{eq:initavail}
Z_f\mathtt{(i)}-Z_r\mathtt{(i)} = \sum_{\mathtt{k=1,NOMBLAY}} E_s\mathtt{(k)}. 
\end{equation}
In general, the thickness of the bed is taken to be large (by default, $100$ m), so
the bottom layer thickness need to be increased.

Assuming the porosity $n$ of each class to be identical and constant, the total mass of sediments per unit area is:

\begin{equation}\label{eq:masssedclass}
M_s\mathtt{(i)} = \sum_{\mathtt{k=1,NOMBLAY}}\rho_s E_s\mathtt{(k)}(n-1). 
\end{equation}

In each layer, the percent of each class \texttt{AVAI}, which is defined as the
volume of each class of material per the total volume of material, is
considered to be a constant. For each layer and at each point of the domain, the following constraints
need to be satisfied:

\begin{equation*}
\sum_{\mathtt{j=1,NSICLA}} \mathtt{AVAI(i,k,j)} = 1\quad \text{and}\quad 0 \leq \mathtt{AVAI(i,k,j)} \leq 1.  
\end{equation*}

\medskip
\begin{bclogo}[couleur=blue!10,arrondi=0.1, logo=\bcinfo]{Keywords}
The initial sediment bed composition is defined by:
\begin{itemize}
\item {\ttfamily NUMBER OF SIZE-CLASSES OF BED MATERIAL}: {\ttfamily NSICLA = 1}, by default
\item {\ttfamily SEDIMENT DIAMETERS}: {\ttfamily FDM = .01}, by default
\item {\ttfamily SETTLING VELOCITIES}: if {\ttfamily XWC} is not given, the subroutine \texttt{vitchu\_sisyphe.f} is used to compute the settling velocity based on the Stokes, Zanke or Van Rijn formulae as a function of the grain size
\item {\ttfamily SHIELDS PARAMETERS}: for multi grain size, 
the Shields parameter needs to be specified for each class. If only one 
value is specified, the shields parameter will be considered constant for all classes. The default option, no Shields parameter given in steering file, is to calculate the Shields parameter as a function of the sand grain diameter, see logical \texttt{CALAC}.
\item {\ttfamily INITIAL FRACTION FOR PARTICULAR SIZE CLASS}: \ttfamily{AVA0 = 1.;0.;0.;...}, by default
\end{itemize}
\end{bclogo}

For more than one-size classes, the user should specify {\ttfamily NSICLA} values for
the mean diameter, initial fraction, etc. For example, for a mixture of two classes:

\medskip
\begin{bclogo}[couleur=blue!2.5,arrondi=0.1, logo=\bccrayon]{Example}
\begin{itemize}
\item {\ttfamily NUMBER OF SIZE-CLASSES OF BED MATERIAL = 2}
\item {\ttfamily SEDIMENT DIAMETERS = 0.5; 0.5}
\item {\ttfamily SETTLING VELOCITIES = 0.10; 0.05}
\item {\ttfamily SHIELDS PARAMETERS = 0.045; 0.01}
\item {\ttfamily INITIAL FRACTION FOR PARTICULAR SIZE CLASS = 0.5; 0.5}
\end{itemize}
\end{bclogo}


\subsection{Sediment transport of sediment mixtures}
The method programmed in \sisyphe for the treatment of mixtures of sediment with variable grain sizes is classical and based on previous models from the literature (as for example the 1D model Sedicoup developed at Sogreah). There are two layer concepts implemented in \sisyphe. The active layer model based on Hirano's concept~\cite{Hirano} (default) and a newly developed continues vertical sorting model (C-VSM) from Merkel~\cite{([XX], 2012), ([XX], 2012)}.

\subsubsection{Active layer and stratum}
Since only the thin upper layer will be transported, the upper layer is
therefore subdivided into a thin \emph{active layer} and an \emph{active
stratum}. The composition of the active layer is used to calculate the composition of
transported material and the intensity of transport rates for each class of
sediment:

\begin{equation}\label{eq:qsclass}
Q_s =\sum_{\mathtt{j=1,NSICLA}} \mathtt{AVA0(j)} Q_s\mathtt{(j)}, 
\end{equation}
where \texttt{AVA0(j)} is the percentage of the class \texttt{j} in the active layer.

The active stratum is the layer that will exchange sediment with the active
layer in order to keep the active layer to a fixed size. When there is a lot
of erosion and when this active stratum becomes too thin, the stratum lying
underneath will be merged with it.

The active layer thickness concept is not well defined and depends on the flow and
sediment transport characteristics~\cite{vanRijn87}. According to Yalin~\cite{Yalin}, 
it is proportional to the sand diameter of the upper layer. 
The active later thickness generally scales with the bed roughness,
which is typically of the order of a few grain diameters for flat beds up to
few centimeters in the case of rippled beds. In the presence of dunes, the active
layer thickness should be half the dune height~\cite{Ribberink}. 

In \sisyphe, the active layer thickness is an additional parameter of the model, noted \texttt{ELAY0}
, which can be estimated by calibration and specified by the
user in the steering file by keyword \texttt{ACTIVE LAYER THICKNESS}. With the
option \texttt{CONSTANT ACTIVE LAYER THICKNESS = NON}, it is possible to use a space and time varying active layer thickness during the simulation. In \sisyphe it is assumed that \texttt{ELAY0 =} $3 D_m$, with $D_m$ the mean diameter of the
upper layer.

The erosion rate is restricted by the total amount of sediments in the
active layer, which therefore acts as a rigid bed. The same methods applied
for rigid beds are applicable for active layer formulation. When dealing
with graded sediment and thin thickness active layers, it is advised to use finite elements combined with the positive depth algorithm, as used for the treatment of rigid bed, in order to avoid numerical problems such as negative sediment fractions. The error message \texttt{time step should be reduced} can also appear in the listing
file when the erosion is greater than half of the active layer thickness. It
is strongly recommended to follow this message.

\subsubsection{Continuous vertical sorting model}

The C-VSM overcomes many limitations of the classic layer implementation of Hirano (HR-VSM).
Even though it is a different way to manage the grain sorting, it is just another interpretation of Hiranos
original idea with fewer simplifications. So still an active layer is defined. But the grain distribution in this layer
is computed each time step from a depth dependent bookkeeping model for each grain size fraction with unlimited numerical
resolution for each horizontal node.
The new model doesn't overcome the need to carefully calibrate the same
input parameters as all other models, but the new interpretation has the following advantages:

\begin{itemize}
\item It is possible to keep minor but prominent grain mixture variations even after a high number of time
steps. Smearing effects and bookkeeping accuracy is defined by user defined thresholds or the
computational resources, rather than through a fix value.

\item Various functions for the impact depth of the shear stress can be chosen to the projects demands.
The result is a much more accurate vertical grain sorting, which results in better prognoses for bed
roughness, bed forms and erosion stability.
\end{itemize}

A couple of new keywords must be set in your sis.cas file.
The full C-VSM output can be found in the new Selafin files VSPRES \& VSPHYD in the tmp-folders. As the higher
resolution of the C-VSM needs resources, you can reduce the printout period, or suppress the output at all. The
common SISYPHE result files only show the averaged values for the active layer. 
Even more disk space can be saved, if only few points are printed out as .VSP.CSV 
files in the subfolder /VSP/. We recommend using between $200$ and $1000$ vertical sections.
More will not improve the accuracy much, and less will lead to increasing data management, as the profile compression
algorithms are called more often. 

\medskip
\begin{bclogo}[couleur=blue!10,arrondi=0.1, logo=\bcinfo]{Keywords}
The initial sediment bed composition is defined by:
\begin{itemize}
\item {\ttfamily VERTICAL GRAIN SORTING MODEL = 1}
\begin{itemize}
\item {\ttfamily 0 = Layer = HR-VSM} (Hirano $+$ Ribberink as until \sisyphe v6p1)
\item {\ttfamily 1 = C-VSM}
\end{itemize}
\item {\ttfamily C-VSM MAXIMUM SECTIONS = 100}
\begin{itemize}
\item Should be at least $4 + 4 \times$ {\ttfamily Number of fractions},
\item better $> 100$, tested up to $10000$
\end{itemize}
\item {\ttfamily C-VSM FULL PRINTOUT PERIOD = 1000}
\begin{itemize}
\item {\ttfamily 0 =>  GRAPHIC PRINTOUT PERIOD}
\item Anything greater {\ttfamily 0 =>} Sets an own printout period for the {\ttfamily CVSP}
\item Useful to save disk space
\end{itemize}
\item {\ttfamily C-VSM PRINTOUT SELECTION = 0|251|3514|...|...}
\begin{itemize}
\item Add any 2D Mesh Point numbers for {\ttfamily.CSV-Ascii} output of the CVSP
\item Add {\ttfamily 0} for full {\ttfamily CVSP} output as Selafin3D files (called {\ttfamily VSPRES $+$ VSPHYD})
\item All files are saved to your working folder and in {\ttfamily /VSP} \& {\ttfamily /LAY} folders below
\end{itemize}
\item {\ttfamily C-VSM DYNAMIC ALT MODEL = 5}
\begin{itemize}
\item Model for dynamic active layer thickness approximation
\item {\ttfamily 0} = constant (Use Keyword: {\ttfamily ACTIVE LAYER THICKNESS})
\item {\ttfamily 1} = Hunziker \& Guenther
      \begin{equation*}
          ALT = 5 d_{MAX}
       \end{equation*}          
\item {\ttfamily 2} = Fredsoe \& Deigaard (1992)
       \begin{equation*}
        ALT =\frac{2 \tau_B}{(1-n) g (\rho_S - \rho) \tan \Phi} 
       \end{equation*}
\item {\ttfamily 3} = van RIJN (1993)
       \begin{equation*}
        ALT =0.3 D_*^{0.7}\frac{\tau_B-\tau_C}{\tau_B}^{0.5} d_{50} 
       \end{equation*}
\item {\ttfamily 4} = Wong (2006)
       \begin{equation*}
        ALT =5 \frac{\tau_B}{(\rho_S - \rho) g d} - 0.0549)^{0.56} d_{50} 
       \end{equation*}
\item {\ttfamily 5} = Malcharek (2003)
       \begin{equation*}
        ALT =\frac{d_{90}}{1-n} \max(1,\frac{\tau_B}{\tau_C}) 
       \end{equation*}
\item {\ttfamily 6 = 3*d50} within last time steps ALT'
       \begin{equation*}
        ALT =3 d_{50} 
       \end{equation*}
\end{itemize}
\end{itemize}
\end{bclogo}

\subsubsection{General formulation}
Bedload transport rates are computed separately for each class using
classical sediment transport formulae, corrected for sand grading effects. 
The Exner equation is then solved for each class of sediment. The
individual bed evolution due to each class of bed material is then summed
to give the global evolution due to bedload. Similarly, the suspended transport equation is solved for each class of
sediment and the resulting bed evolution for each class is then summed to give
the global evolution due to the suspended load. 

At each time step, the bed evolution due to bedload and to suspension transport is
used to compute the new bed structure. The algorithm which determines the new bed
composition considers two possibilities, namely deposit or erosion, and ensures
mass conservation for each individual class of material. The algorithm is
explained in Gonzales de Linares~\cite{Gonzales}.

\subsubsection{Hiding exposure}
In order to calculate bedload sediment transport rates for each class of
sediment, it is necessary to consider the effect of
hiding exposure. That means that in a sediment mixture, bigger particles will be more exposed to the
flow than the smaller ones and therefore, prediction of sediment transport rates with classical sediment transport
formulas, need to be corrected by use of a \emph{hiding-exposure} correction factor. In \sisyphe, 
the keyword \texttt{HIDING FACTOR FOR PARTICULAR SIZE CLASS} sets the value of hiding factor for a particular size class and the keyword \texttt{HIDING FACTOR FORMULA} allows the user to choose among different formulations. Two formulas, Egiazaroff~\cite{Egiazaroff} and Ashida \& Michiue~\cite{Ashida}, have been written
based on the Meyer-Peter and M\"{u}ller formula. Both formulas modify the critical Shields parameter that will be
used in the Meyer-Peter formula. The formula of Karim and Kennedy~\cite{Karim} can be used in combination with any bedload transport predictor. This formula modifies directly the bedload transport rate.

\begin{itemize}

\item \texttt{HIDFAC = 1}: formulation of Egiazaroff 
\begin{equation*}
\theta_{cr} = 0.047\zeta_i,\quad\text{with}\,\, \zeta_i = \left[\frac{\log (19)}{\log (19 D_i /D_m)} \right]^2 
\end{equation*}
\item \texttt{HIDFAC = 2}: formulation of Ashida \& Michiue 

\item \texttt{HIDFAC = 4}: formulation of Karim \emph{et al.}

\end{itemize}

\medskip
\begin{bclogo}[couleur=blue!10,arrondi=0.1, logo=\bcinfo]{Keywords}
Sand grading effects are defined by:
\begin{itemize}
\item {\ttfamily HIDING FACTOR FORMULA}: if {\ttfamily HIDFAC = 0} (by default), the user needs to give {\ttfamily HIDING FACTOR FOR PARTICULAR SIZE CLASS}
\item {\ttfamily ACTIVE LAYER THICKNESS}: {\ttfamily = 10000}, by default 
\item {\ttfamily CONSTANT ACTIVE LAYER THICKNESS}: {\ttfamily = YES}, by
default
\item {\ttfamily NUMBER OF BED LOAD MODEL LAYERS}: {\ttfamily NOMBLAY = 2}, by
default
\end{itemize}
\end{bclogo}

\subsubsection{Sediment transport formula}
The formula of Hunziker~\cite{Hunziker} is an adaptation of the Meyer-Peter M\"{u}ller formula for fractional transport. The volumetric sediment transport per sediment class is given by:
\begin{equation}\label{eq:Hunziker}
\Phi_b  = 5 p_i \left[\zeta_i\left( \mu \theta_i -\theta_{cm} \right) \right]^{3/2} \quad\text{if}\quad \mu \theta_i > \theta_{cr}, 
\end{equation}
with $p_i$ the fraction of class $i$ in the active layer, $\theta_i$ the Shields parameter, 
$\theta_{cm} = \theta_{cr}\left(D_{mo}/D_m\right)^{0.33}$ the corrected critical Shields parameter, $\theta_{cr}$ the critical Shields parameter ($\theta_{cr}=0.047$), $\xi_{i} =\left(D_i/D_m\right)^{-\alpha}$ the hiding factor, $D_i$ the grain size of class $i$, $D_m$ the mean grain size of the surface layer (m), $D_{mo}$ the mean grain size of the under layer (m), and $\alpha$ a constant. The critical Shields parameter is calculated as a function of the
dimensionless mean grain size $D_*$. It should be noted that according to
Hunziker, stability problems may occur outside the parameter range $\alpha \leq 2.3$ and $D_i/D_m \geq 0.25$.





