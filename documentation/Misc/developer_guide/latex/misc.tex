%
%%%%%%%%%%%%%%%%%%%%%%%%%%%%%%%%%%%%%%%%%%%%%%%%%%%%%%%%%%%%%%%%%%%%%%%%
\chapter{Usefull stuff}
%%%%%%%%%%%%%%%%%%%%%%%%%%%%%%%%%%%%%%%%%%%%%%%%%%%%%%%%%%%%%%%%%%%%%%%%
%
\section{Little script to check part of the coding convention}
%
In the trunk version or after (V7.0) you can find a script called
\verb!check_code.sh! that will scan your source code and check for a few points
of the coding convention. You should run this script before submitting your
development. Below is the description the script will give you if you launch it
with the \verb!-h! option. This script is located under the \verb!scripts!
folder in the \telemacsystem directory.
\begin{lstlisting}
Usage: check_code.sh source_path
Script checking some points of the coding convention 
for all the .f and .F in the folder given in parameter
It will generate 4 files:
-- indent.log : contains the line where the indentation is not a 6 + x*2
-- comments.log : checks that the character used for comments is '!'
-- continuation.log : checks that the character for continuation is '&'
-- lowercase.log : checks that there are no lowercase code
\end{lstlisting}

\section{Adding a new test case}
%
This section will guide you through the hard but needed action of adding a new
case do not frighten it is not that hard. First of all you need to create a new
folder in the examples in the folder corresponding to the module the test case
is for. That folder must contain the following elements:
\begin{itemize}
\item All the \textbf{input} files you need to run the case, and just the input
the files generated by a successful run should not be in the SVN repository.
\item A reference file to run the validation.
\item The \verb!doc! folder which contains the documentation for the test case
in LaTeX format (See other test cases for example).
\item The xml file to run the test case (See other test cases for example).
\end{itemize}
