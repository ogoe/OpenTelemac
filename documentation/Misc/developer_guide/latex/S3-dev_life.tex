\chapter{Development life in Telemac}
\label{ref:devlife}
Greetings fellow \telemacsystem developer and welcome into the world of a \telemacsystem
developer. It might be hard at the beginning but with time you will unravel all
of \telemacsystem dirty little secrets.
%
This Guide aims to describe all the steps you might encounter when developing
in \telemacsystem, those steps can be found in the "\telemacsystem Software Qualtiy Plan"
(Eureka H-P74-2014-02365-EN). They are resumed in the Fig \ref{cycle}.
%
\begin{figure}[H]
\centering
\begin{tikzpicture}[node distance = 1cm, auto]
  % Defining style for the task
  \tikzstyle{task} = [draw, very thick, fill=white, rectangle]
  \tikzstyle{bigbox} = [draw, thin, rounded corners, rectangle]
  \tikzstyle{box} = [draw, thick, fill=white, rounded corners, rectangle]
  % Creation of the nodes
  \node (CUE) [task] {1. Ticket on CUE};
  \node (CHs) [task, below of=CUE] {2. CHs};
  \node (ADEPHE) [task, below of=CHs] {3. ADEPHE};
  \node (IMPL) [task, below of=ADEPHE] {4. Implementation};
  \node (null2) [right of=IMPL, node distance=9em] {};
  \node (VV) [task, below of=IMPL] {5. Verification \& Validation};
  \node (DOC) [task, below of=VV] {6. Documentation};
  \node (INT) [task, below of=DOC] {7. Integration "ADEPHE"};
  \node (null1) [right of=INT, node distance=9em] {};
  \node (null3) [left of=INT, node distance=9em] {};
  \node (TRUNK) [box, below of=INT, node distance=4em] {Main branch of \telemacsystem};
  \node (VALID) [task, below of=TRUNK] {Full validation};
  \node (TAG) [box, below of=VALID] {Release of \telemacsystem (tag)};
  % big box
  \node (DEV) [bigbox, fit = (CUE) (CHs) (ADEPHE) (IMPL) (null1) (null2) (null3) (VV) (DOC) (INT)] {};
  \node at (DEV.north) [above, inner sep=3mm] {\textbf{Development}};
  % Creation of the path between the nodes
  \draw[->] (CUE) to node {} (CHs);
  \draw[->] (CHs) to node {} (ADEPHE);
  \draw[->] (ADEPHE) to node {} (IMPL);
  \draw[->] (IMPL) to node {} (VV);
  \draw[->] (VV) to node {} (DOC);
  \draw[->] (DOC) to node {} (INT);
  \draw[-] (INT) -- node [near start] {no} (null1.center);
  \draw[-] (null1.center) -- (null2.center);
  \draw[->] (null2.center) -- node [near start] {} (IMPL);
  \draw[->] (INT) to node {yes} (TRUNK);
  \draw[->] (TRUNK) to node {} (VALID);
  \draw[->] (VALID) to node {} (TAG);
\end{tikzpicture}
\caption{\label{cycle}Life cycle of a \telemacsystem development}
\end{figure}
%
The following sections will describes how to use the three tools you will be
using during your development:
\begin{itemize}
\item SVN, The source controller that you will be using to handle the sources of
your development.
\item CUE, The ticket manager which will contain information on your
development.
\item CIS, The Continuous Integration Service which will compile, run, and
validation your developments.
\end{itemize}
%
\label{mail}
%
%-----------------------------------------------------------------------
\section{Information to give when your development is beginning}
%-----------------------------------------------------------------------
%
To begin you work you will need to send an email to \url{awe@hrwallingford.com}
with the following informations in order to give you the proper access into the
\telemacsystem developing world:
\begin{itemize}
\item Your Name.
\item The name of a fish (for the SVN branch, any existing fish is okay).
\item The \telemacsystem modules on which you will be working (See Section
  \ref{proj}).
\item The demand to add your branch on CIS.
\end{itemize}
%
%-----------------------------------------------------------------------
\section{Information to give when your development is over}
%-----------------------------------------------------------------------
%
This section will give a list of all the information you need to give to the
person in charge of the integration of your development.
%
\begin{itemize}
\item CUE ticket number.
\item Name of the branch the work is done on.
\item Revision range of work I.e. The revision number of when you started your
  development and when it ended on your SVN branch.
\item A list of the file impacted by the development (See \ref{diff} on how to
create that list).
\item The name of the test cases validating the development (\ref{testcase} on
  how to add a new test case).
\end{itemize}

All those information should be written in the CUE ticket as well.
