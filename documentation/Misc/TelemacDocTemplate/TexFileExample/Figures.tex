%-------------------------------------------------------------------------------
\chapter{Adding figures}
%-------------------------------------------------------------------------------

%-------------------------------------------------------------------------------
\section[Short section title]{Adding a figure in your text so that it will not move}
%-------------------------------------------------------------------------------

It is recommended to store you figures in a seperate ``Figures'' folder that
will contain all the figures of the manual. Furthermore PDFLaTex or XeLaTex
allow you to use .png, .jpg or .pdf figures. If you want to use .ps or .eps
figures you are going to need to use LaTeX or XeTeX, but it is not recommended
for the \telemacsystem{} documentation.

We distinguish two type of figures the one generated by the validation and the
others. For the one generated by the validation they should not be added to svn
repository and when including then use the following syntax (mainly the same as
other figure just use \verb!\includegraphicsmaybe! instead of
\verb!\includegraphics!):
\begin{verbatim}
%
\includegraphicsmaybe{[width=0.7\textwidth]}{./FigExample/generatedFigure}
%
\end{verbatim}
%
\begin{WarningBlock}{Warning:}
That function only works for \LaTeX file that are inside the doc folder in the
validation examples.
\end{WarningBlock}

For the others follow the indication below

Here is example of a non-floating figure:

\begin{figure}[H]%
\begin{center}
%
  \includegraphics[width=0.7\textwidth]{./FigExample/ExampleImage}
%
\end{center}
\caption
[Short caption]
{Long caption of a figure consisting of Saint-Venant Laboratory in french and
saved as a png\protect\footnotemark.}
\label{fig:ExampleImage}
\end{figure}
\footnotetext[1]{Here is how to add a footnote within a caption.}

The figure is referenced as figure~\ref{fig:ExampleImage}. If you want your
figures to float use the options ``[ht!]'' or ``[htb!]'' after your
\verb+\begin{figure}+, but it is recommended to use the option ``[H]'' for the
\telemacsystem{} documentation.

%-------------------------------------------------------------------------------
\section{Having multiple figures}
%-------------------------------------------------------------------------------

It is possible to have multiple figures with captions, see
figure~\ref{fig:ExampleMultipleImages}.

\begin{figure}[H]%
\begin{center}
%
\hfil
%
\subfloat[Logo in french]{
%
  \includegraphics[width=0.45\textwidth]{./FigExample/ExampleImage}
%
}
%
\hfil
%
\subfloat[Logo in english]{
%
  \includegraphics[width=0.45\textwidth]{./FigExample/ExampleImage}
%
}
%
\hfil
\mbox{}
\end{center}
\caption
[Different Logos for LHSV]
{Different Logos for LHSV.}
\label{fig:ExampleMultipleImages}
\end{figure}

%-------------------------------------------------------------------------------
\section{Plotting more complicated graphics}
%-------------------------------------------------------------------------------

Some powerful packages exist to plot high quality vectorial graphics in LaTeX.
It is recommended to use the packages TiKZ and pgfplots, and to have one *.tex
file per image.
