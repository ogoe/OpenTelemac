


\chapter{ Secondary currents}

 In a curved channel, the flow experiences a radial acceleration and centrifugal forces act in proportion to the mean velocity.  
In turn, the surface of the water is tilted radially on the outer bank to produce a super-elevation sufficient to create a pressure gradient 
to balance the average centrifugal force.  At shallower depths, the centrifugal force exceeds the pressure force, whence the resultant force 
drives the fluid outwards.

TELEMAC-2D allows to take into account the effect of these secondary currents. To activate this, key-word \textit{SECONDARY CURRENTS} must be set to \textit{TRUE} 
(default \textit{FALSE}). User can also manage some coefficients used in the resolved equation (see release note 7.0 for theoretical aspects and 
for more details). For instance, the production term in the advection-diffusion equation depends linearly on a coefficient As which can be 
calibrated using key-word\textit{ PRODUCTION COEFFICIENT FOR SECONDARY CURRENTS} (default 7.071).  In the same way, 
the dissipation part can be modified by varying the coefficient $A_{ds}$ using key-word \textit{ DISSIPATION COEFFICIENT FOR SECONDARY CURRENTS} 
(default 0.5).

 An example of use of the secondary currents is given in the validation case SECCURRENTS (in folder /examples/telemac2d). 
This example is described and well documented in the validation manual of TELEMAC-2D.

