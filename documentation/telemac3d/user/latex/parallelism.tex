\chapter{Parallelism}

For simulations requiring a high computational power, it can be advisable to
run the computations in multi-processor machines, or in clusters of
workstations. \telemac{3D} is available in a parallel version in order to take
advantage of that kind of computational architecture.

The \telemac{3D} parallel version uses the MPI library which has to be installed
beforehand to be implementable. The interface between \telemac{3D} and that MPI
library is achieved through the parallelparallel library which is common to all
the TELEMAC system modules.

Lots of pieces of information concerning the implementation of the parallel
version can be found in the system's installation literature.

The user shall initially specify the number of processors used by means of the
keyword \telkey{PARALLEL PROCESSORS}. That integer type keyword can assume the
following values:

\begin{itemize}
\item  0: Implementation of the conventional \telemac{3D} version (default),

\item  1: Implementation of the parallel \telemac{3D} version on a processor,

\item  2 ...: Implementation of the parallel \telemac{3D} version using the
specified number of processors.
\end{itemize}
