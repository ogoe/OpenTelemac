\section{Manche}
%
% - Purpose & Problem description:
%     These first two parts give reader short details about the test case,
%     the physical phenomena involved and specify how the numerical solution will be validated
%
\subsection{Purpose}
%
This test case has been exhibited from an old version (V5P4), the reference case at this time was called . It has got mainly one interest, to compare results from old versions to the new version.  

%
\subsection{Description of the problem}
We simulate the storm that occured in 1990 in Manche. This test was a test described in the validation of tomawac 1.0. Because of many changes in the code it had been abandonned. It now works with the new version but not in parallel because of the fortran user that sets the boundary. 

We discretize with 25 frequencies and 12 directions. The minimal frequency is 0.04177248 and the frequential ratio 1.1.
The mesh is shown on picture \ref{figmanchemesh}, as we are interested in storm in the Manche the mesh is finer around the manche. 
The time step is 300s and we simulate during 14 days. 
\begin{figure} [!h]
\centering
\includegraphicsmaybe{[width=0.85\textwidth]}{../img/mesh.png}
 \caption{Mesh of the sea. }
\label{figmanchemesh}
\end{figure}

In fact we are going to work with spherical coordinate so the 'real' bathymetry will be the one showed on Figure \ref{realbathymanche}. In the code if the bathymetry is more than -10 m, it is set to -10 m.
\begin{figure} [!h]
\centering
\includegraphicsmaybe{[width=0.85\textwidth]}{../img/bathy.png}
 \caption{Real Bathymetry used in the code.}
\label{realbathymanche}
\end{figure}

We take into account the wind through a file that gives the wind in all point of the mesh every 6 hours. On this point, in the original file the time was given with a deprecated format, so we took the wind from the results of the old simulation which was given in a new format (number of seconds after 0).

Nowadays time format is given through a number \textit{YYYYMMDDHHMMNN}, and it used to be given by MDDHH. 
As we kept the fortran user to describe boundaries conditions one will read this format in the keyword \textit{DATE DE DEBUT DU CALCUL = 1161500} for \textit{1990 01 16 15:00:00}

\subsection{Results and Comments}

If one run the validation, one will see a difference of 30\% wave heigth result, it is only on a point and it doesn spread on all the result, so we guess it is a change in the postprocessing. It could be explained by the fact that may be integration over directions and frequencies used to be made in simple precision.
When we observe global results, we would say that there are no difference between reference and new result (see \ref{figmanchehm0} \ref{figmanchehm02} and \ref{figmanchehm0v1P3}.

As the calculation is made in spherical coordinates but the initial mesh is done in cartesian coordinates. Tomawac calculate a conversion through Fortran user. The result is a little bit different if we take the conversion from the coordinate of the old version because of simple precision in the coordinate. That is the reason of the difference between last comparison when running validation.  

\begin{figure} [!h]
\centering
\includegraphicsmaybe{[width=0.85\textwidth]}{../img/results.png}
 \caption{Wave height for last version}
\label{figmanchehm0}
\end{figure}
\begin{figure} [!h]
\centering
\includegraphicsmaybe{[width=0.85\textwidth]}{../img/results2.png}
 \caption{Wave height for coordinate changed outside the code.}
\label{figmanchehm02}
\end{figure}
\begin{figure} [!h]
\centering
\includegraphicsmaybe{[width=0.85\textwidth]}{../img/resultsV1P3.png}
 \caption{Wave height for V1P3 version}
\label{figmanchehm0v1P3}
\end{figure}
\begin{figure} [!h]
\centering
\includegraphicsmaybe{[width=0.85\textwidth]}{../img/direction.png}
 \caption{Mean Direction for last version}
\label{figmanchedirection}
\end{figure}
\begin{figure} [!h]
\centering
\includegraphicsmaybe{[width=0.85\textwidth]}{../img/directionref.png}
 \caption{Mean direction for reference file}
\label{figmanchehm0}
\end{figure}
