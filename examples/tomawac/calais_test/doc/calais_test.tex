\section{Calais}
%
% - Purpose & Problem description:
%     These first two parts give reader short details about the test case,
%     the physical phenomena involved and specify how the numerical solution will be validated
%
\subsection{Purpose}
%
This test case has been exhibited from an old version. It has got mainly one interest, to compare results from version 5P8 to the new version.  

%
\subsection{Description of the problem}
We simulate waves around the harbour of Calais coming from offshore. We neglect the influence of people trying to accross the channel.
At the boundary, the significative heigth is 4m the frequency peak 0.1 and the direction of 135 degree.

We discretize with 23 frequencies and 24 directions. The mesh is shown on puicture \ref{figcalaismesh}

\begin{figure} [!h]
\centering
\includegraphicsmaybe{[width=0.85\textwidth]}{../img/mesh.png}
 \caption{Mesh of the neighborood of Calais. }
\label{figcalaismesh}
\end{figure}

\subsection{Results and Comments}

If one run the validation, he will see a diffrence of 11 degree on direction in results, 11 Watt on Power and 0.5 m on wave heigth. But as we can see on figure  \ref{figcalaishm0},\ref{figcalaishm0v6p0}, \ref{figcalaishm0v5p8},this is not the case when observing global results.

\begin{figure} [!h]
\centering
\includegraphicsmaybe{[width=0.85\textwidth]}{../img/hm0.png}
 \caption{Wave height for last version}
\label{figcalaishm0}
\end{figure}
\begin{figure} [!h]
\centering
\includegraphicsmaybe{[width=0.85\textwidth]}{../img/hm0v6p0.png}
 \caption{Wave height for V6p0 version}
\label{figcalaishm0v6p0}
\end{figure}
\begin{figure} [!h]
\centering
\includegraphicsmaybe{[width=0.85\textwidth]}{../img/hm0v5p8.png}
 \caption{Wave height for V5p8 version}
\label{figcalaishm0v5p8}
\end{figure}
\begin{figure} [!h]
\centering
\includegraphicsmaybe{[width=0.85\textwidth]}{../img/direction.png}
 \caption{Direction for last version}
\label{figcalaisdirection}
\end{figure}
\begin{figure} [!h]
\centering
\includegraphicsmaybe{[width=0.85\textwidth]}{../img/directionv6p0.png}
 \caption{Direction for V6P0  version}
\label{figcalaishm0}
\end{figure}
\begin{figure} [!h]
\centering
\includegraphicsmaybe{[width=0.85\textwidth]}{../img/power.png}
 \caption{Power for last version}
\label{figcalaishm0}
\end{figure}
\begin{figure} [!h]
\centering
\includegraphicsmaybe{[width=0.85\textwidth]}{../img/powerv5p8.png}
 \caption{Direction for V5P8 version}
\label{figcalaishm0}
\end{figure}
