\subsection{Yen}
%

% - Purpose & Problem description:
%     These first two parts give reader short details about the test case,
%     the physical phenomena involved and specify how the numerical solution will be validated
%
\subsubsection{Purpose}
%
The purpose of this test is to assess the accuracy of Sisyphe at reproducing the bed evolution in an alluvial channel bend under 
unsteady-flow conditions. The mechanics of sediment transport in channel bends, frequently appearing in natural rivers, are much
 more complex than that in straight channels. The complexity is twofold. On the one hand, the sediment transport in a channel bend 
is subject not only to longitudinal transport but also to transverse transport and transverse sorting by the secondary flow 
inherently associated with bends. On the other hand, the unsteadiness of flow in natural rivers certainly has some effects on the 
structure of the flow field, thereby affecting the motion of sediment particles.

This test is the experimental setup (RUN 5) proposed by Yen and Lee (1995). In this case, the bed evolution of a 180$^o$ channel 
bed with an initial flat bottom is computed for a triangular-shaped $300$ min. hydrograph. Numerical results are validated by 
measured contours of bed evolution after at the end of the experience and by measured bottom elevations at
two different cross sections (90$^o$ and 180$^o$). Although in the original experience of Yen and Lee the sediment was slightly 
graded. This validation case is for graded sediment distribution. 
%
\subsubsection{Problem setup}
%
The flume consists of a straight section of $11.5$ m
long, a $180^o$ bend of $4.0$ m radius and a downstream straight section of
$11.5$ m long, with a constant slope in flow direction equal to $0.002$. The width of the flume channel is $1.0$ m. A 
triangular-shaped inflow hydrograph with an initial discharge of $Q=0.02$ m$^3/$s, a water depth at the outflow of $h = 0.0544$ m 
and a peak discharge of equal to $0.053$ m$^3/$s (water depth $h=0.103$m) at $T = 100$ min
is used, see Figure~\ref{fig:hydro}. After $T = 100$ min, the inflow discharge is reduced linearly until it reached the 
initial values at the end of the experiment ($T = 300$ min).

The sediment is characterized by an median diameter of $D_{50}=1$ mm. Five sediment classes 
($D_1=0,31$, $D_2=0,64$, $D_3=1,03$, $D_4=1,69$, $D_5=3,36$ each 20 \%) are chosen in order to adopt the sediment distribution 
of the experiment. 
 The Engelund-Hansen formula is used to estimate the sediment transport capacity of the channel. The slope effect and the 
secondary currents correction are accounted for this test. The influence of the slope effect on the the direction of the bedload 
transport is accounted through the Talmon formula, with $\beta_2=0.85$. 
The influence of the slope effect on the the magnitude of the bedload transport is accounted through the Soulsby formula, with 
an friction angle of 35$^o$
The default value of $\alpha=1$ is used for the secondary currents parameter, therefore the Engelund parameter $A=7$. 
Two vertical sediment layers with a sum thickness of 20 cm are assumed.

\begin{figure} [!h]
\centering
\includegraphics[scale=0.15, bb=0 0 30 30]{img/yen_boundaries.png}
 \caption{Triangular-shaped hydrograph.}\label{fig:hydro}
\end{figure}

A friction closure relationship, based on the Nikuradse roughness length is adopted to
account for the bed resistance. For this case, $k_s=3.5$ mm ($\approx 3\times D_{50}$) and the Elder model is specified to 
parameterize the turbulent eddy viscosity. 
The critical Shields parameter is set at $0.047$ and the bed porosity is $0.375$.


% - Reference:
%     This part gives the reference solution we are comparing to and
%     explicits the analytical solution when available;
%
%
\subsubsection{Numerical setup}
%
Numerical simulations were conducted on a un-
structured, triangular finite element mesh with
$3230$ elements and $1799$ nodes and a mean grid size of $[0.20]$ m (Figure~\ref{fig:mesh}). 
As initial condition, a fully developed (stationary) flow with a constant water-depth $h = 0.0544$m and discharge $0.02$ 
m$^3/$s is imposed and the bottom has a constant slope in flow direction of 0.002. 

\begin{figure} [!h]
\centering
\includegraphics[scale=0.15, bb=0 0 100 100]{img/yen_grid_bottom.png}
 \caption{Finite element discretization of the bend.}\label{fig:mesh}
\end{figure}

The time step is set to $0.5$ s. For a mean velocity in the range $[0.37-0.53]$ m/s and 
a mean grid size of the order of $0.2$ m, the mean Courant number varies between $0.6-1,3$. 

\subsubsection{Results}
%
Numerical results of the normalized bed evolution are shown in Figure~\ref{fig:results1}. Morphological changes exhibit the expected
patterns of erosion and sedimentation at the channel bend, with the presence of a point bar along the inner-bank and a deeper 
channel along the outer-bank of the bend. A python script compares the measured isolines with the simulated coloured contour plot 
of the normalized bed evolution. 
The computed bed changes are in agreement with the measured data. Without accounting for the secondary flow effect, one cannot 
obtain such reasonable results. 

The bottom evolution in the two cross sections 90 ° and 180 ° can be compared with measurements
after a simulation time of 5 hours. 
A python script compared measurements and simulation at both cross sections (see Figure~\ref{fig:results2}).

During the validation process the results of the serial run were compared with the measurements and the figures can be found in the directories
CompareResults_CrossSections and CompareResults_Evolution. 

\begin{figure} [!h]
\centering
\includegraphics[scale=0.15, bb=0 0 30 30]{img/EvolutionR05.png}
 \caption{Comparison of simulated (coloured) and measured (black contour lines) normalized bed evolution.}\label{fig:results1}
\end{figure}

\begin{figure} [!h]
\centering
\includegraphics[scale=0.5, bb=0 0 30 30]{img/sis_yen-exp_90.png}
\includegraphics[scale=0.5, bb=0 0 30 30]{img/sis_yen-exp_180.png}
 \caption{Comparison of simulated and measurement bottom elevation at cross section 90 ° and 180 °.}\label{fig:results2}
\end{figure}
% Here is an example of how to include the graph generated by validateTELEMAC.py
% They should be in test_case/img



\subsubsection{References}
%
Yen, C. and Lee, K.T. (1995) \textit{ Bed Topography and Sediment Sorting in Channel Bend 
with Unsteady Flow}. Journal of Hydraulic Engineering, Vol.121, No. 8.
