% case name
\chapter{solit}
%
% - Purpose & Description:
%     These first two parts give reader short details about the test case,
%     the physical phenomena involved, the geometry and specify how the numerical solution will be validated
%
\section{Purpose}
%
This test demonstrates the ability of \telemac{3d} to model the
propagation of a solitary wave with only 2, 3 or 4 vertical levels.
This study demonstrates also the necessity of using the non-hydrostatic
version of the software.
%
\section{Description}
%
We consider a flat channel 600~m long and 6~m wide with a constant water
depth of 10~m with no velocity.
A solitary wave is defined at the left part of the channel with
theoretical initial velocity.
We observe the propagation of this wave according to various numbers of
planes on the vertical.
%
% - Reference:
%     This part gives the reference solution we are comparing to and
%     explicits the analytical solution when available;
%
% bibliography can be here or at the end
%\subsection{Reference}
%
%
\subsection{Reference}
%

%
% - Geometry and Mesh:
%     This part describes the mesh used in the computation
%
%
\subsection{Geometry and Mesh}
%
\subsubsection{Bathymetry}
%
Flat bottom ($z$ = -10~m)
%
\subsubsection{Geometry}
%
Channel length = 600~m\\
Channel width = 6~m
%
\subsubsection{Mesh}
%
7,206 triangular elements\\
4,210 nodes\\
2, 3 or 4 planes regularly spaced
%
% - Physical parameters:
%     This part specifies the physical parameters
%
%
\subsection{Physical parameters}
%
Diffusion: no\\
Coriolis: no\\
Wind: no
%
% Experimental results (if needed)
%\subsection{Experimental results}
%
% bibliography can be here or at the end
%\subsection{Reference}
%
% Section for computational options
%\section{Computational options}
%
% - Initial and boundary conditions:
%     This part details both initial and boundary conditions used to simulate the case
%
%
\subsection{Initial and Boundary Conditions}
%
\subsubsection{Initial conditions}
%
Constant water depth = 10~m\\
Initial wave centred at $x$ = 150~m\\
Initial theoretical velocity of the wave
%
\subsubsection{Boundary conditions}
%
Closed boundaries. No bottom friction
%
\subsection{General parameters}
%
Time step: 0.1~s\\
Simulation duration: 30~s
%
% - Numerical parameters:
%     This part is used to specify the numerical parameters used
%     (adaptive time step, mass-lumping when necessary...)
%
%
\subsection{Numerical parameters}
%
Non-hydrostatic version\\
Advection of velocities: Characteristics
%
\subsection{Comments}
%
% - Results:
%     We comment in this part the numerical results against the reference ones,
%     giving understanding keys and making assumptions when necessary.
%
%
\section{Results}
%
Figure 3.11.1 presents a longitudinal cross profile of the free surface
at different times of the simulation and for the various configurations
on vertical discretisation.
The amplitude of the wave remains constant in all cases.
However a little decrease of the wave height can be observed on the
simulation with only 2 planes.
The theoretical velocity of the propagation is equal to $\sqrt{gh}$.
Taking into account the value of $h$ (10~m), the wave height (1~m),
the velocity of the top of the wave is equal to 10.38~m/s.
For duration of 30~s, the displacement is equal to 311~m.
The position of the wave at the end of the simulation computed by
\telemac{3d} has a good agreement with the theory.
%
\section{Conclusion}
%
\telemac{3d} simulates correctly the propagation of a solitary wave.
%
% Here is an example of how to include the graph generated by validateTELEMAC.py
% They should be in test_case/img
%\begin{figure} [!h]
%\centering
%\includegraphics[scale=0.3]{../img/mygraph.png}
% \caption{mycaption}\label{mylabel}
%\end{figure}
%
% bibliography
%\section{Reference}
