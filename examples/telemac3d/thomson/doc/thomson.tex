% case name
\chapter{thomson}
%
% - Purpose & Description:
%     These first two parts give reader short details about the test case,
%     the physical phenomena involved, the geometry and specify how the numerical solution will be validated
%
\section{Purpose}
%
This test is identical to case 3.3 but with open boundaries.
It shows that the circular wave propagates out of the computational
domain freely without any reflexion.
It demonstrates that the \telemac{3d} solution is not polarised because
it can simulate the circular spreading of a wave in square computation
domain.
Moreover, it also demonstrates that \telemac{3d} is capable to deal
with open boundaries without prescribing water depth or velocity by
using the Thompson method based on characteristic.
%
\section{Description}
%
The fluid is initially at rest with a Gaussian free surface in the
centre of a square domain (see figure 3.14.2).\\
The evolution of the surface is then calculated during 4 seconds.
%
% - Reference:
%     This part gives the reference solution we are comparing to and
%     explicits the analytical solution when available;
%
% bibliography can be here or at the end
%\subsection{Reference}
%
%
\subsection{Reference}
%

%
% - Geometry and Mesh:
%     This part describes the mesh used in the computation
%
%
\subsection{Geometry and Mesh}
%
\subsubsection{Bathymetry}
%
Flat bottom
%
\subsubsection{Geometry}
%
Square length = 20.1~m
%
\subsubsection{Mesh}
%
8,978 triangular elements\\
4,624 nodes\\
3 planes regularly spaced on the vertical (see figure 3.14.1)
%
% - Physical parameters:
%     This part specifies the physical parameters
%
%
\subsection{Physical parameters}
%
Turbulence: constant viscosity in both directions (molecular viscosity)\\
Bottom friction: Chézy law with coefficient equal to 60~m$^{1/2}$/s\\
Coriolis: no\\
Wind: no
%
% Experimental results (if needed)
%\subsection{Experimental results}
%
% bibliography can be here or at the end
%\subsection{Reference}
%
% Section for computational options
%\section{Computational options}
%
% - Initial and boundary conditions:
%     This part details both initial and boundary conditions used to simulate the case
%
%
\subsection{Initial and Boundary Conditions}
%
\subsubsection{Initial conditions}
%
Water depth at boundary: 2.4~m\\
Water depth at the centre: 4.8~m\\
No velocity
%
\subsubsection{Boundary conditions}
%
Open boundary with the Thompson method based on characteristic
%
\subsection{General parameters}
%
Time step: 0.04 s\\
Simulation duration: 4 s
%
% - Numerical parameters:
%     This part is used to specify the numerical parameters used
%     (adaptive time step, mass-lumping when necessary...)
%
%
\subsection{Numerical parameters}
%
Non-hydrostatic computation\\
Advection for velocities: PSI-type MURD scheme\\
Free surface gradient compatibility coefficient: 0.
%
\subsection{Comments}
%
The initial free surface elevation is prescribed in the
\telkey{CONDIM} subroutine
%
% - Results:
%     We comment in this part the numerical results against the reference ones,
%     giving understanding keys and making assumptions when necessary.
%
%
\section{Results}
%
Figures 3.14.3 to 3.14.5 show that the wave spreads circularly around
the initial water surface peak elevation when it reaches the boundaries,
the wave goes out of the domain freely, no reflection occurs.
%
\section{Conclusion}
%
Even though the mesh is polarised (along the $x$ and $y$ directions and
the main diagonal), the solution is not.\\
Open boundaries are treated properly: no bias occurs.
Moreover, using the Thompson methods method based on characteristic
enable to use open boundaries (444) without the need to specify water
depth or velocity at the boundary.
%
% Here is an example of how to include the graph generated by validateTELEMAC.py
% They should be in test_case/img
%\begin{figure} [!h]
%\centering
%\includegraphics[scale=0.3]{../img/mygraph.png}
% \caption{mycaption}\label{mylabel}
%\end{figure}
%
% bibliography
%\section{Reference}
