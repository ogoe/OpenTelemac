% case name
\chapter{source}
%
% - Purpose & Description:
%     These first two parts give reader short details about the test case,
%     the physical phenomena involved, the geometry and specify how the numerical solution will be validated
%
\section{Purpose}
%
This test shows the capability of \telemac{3d} to manage multiple
sources of fluid and tracers.
It also demonstrates the ability to compute injection and conservation
of multiple tracers.
%
\section{Description}
%
We consider a basin at rest.
Sources are specified at some points of the mesh.
%
% - Reference:
%     This part gives the reference solution we are comparing to and
%     explicits the analytical solution when available;
%
% bibliography can be here or at the end
%\subsection{Reference}
%
%
\subsection{Reference}
%

%
% - Geometry and Mesh:
%     This part describes the mesh used in the computation
%
%
\subsection{Geometry and Mesh}
%
\subsubsection{Bathymetry}
%
Flat bottom ($z$ = -1~m)\\
Constant water depth = 1.0~m
%
\subsubsection{Geometry}
%
Basin length = 100~m\\
Basin width = 40~m (figure 3.7.1)
%
\subsubsection{Mesh}
%
674 triangular elements\\
373 nodes\\
5 layers regularly spaced on the vertical
%
% - Physical parameters:
%     This part specifies the physical parameters
%
%
\subsection{Physical parameters}
%
Constant vertical and horizontal viscosities: 10$^{-6}$ m$^2$/s\\
Coriolis: no\\
Wind: no
%
% Experimental results (if needed)
%\subsection{Experimental results}
%
% bibliography can be here or at the end
%\subsection{Reference}
%
% Section for computational options
%\section{Computational options}
%
% - Initial and boundary conditions:
%     This part details both initial and boundary conditions used to simulate the case
%
%
\subsection{Initial and Boundary Conditions}
%
\subsubsection{Initial conditions}
%
No velocity\\
Null water level\\
No tracers
%
\subsubsection{Boundary conditions}
%
Channel banks: solid boundary without roughness\\
Bottom: solid boundary roughness
%
\subsection{General parameters}
%
Time step: 1.1~s\\
Simulation duration: 1,100~s
%
% - Numerical parameters:
%     This part is used to specify the numerical parameters used
%     (adaptive time step, mass-lumping when necessary...)
%
%
\subsection{Numerical parameters}
%
Hydrostatic simulation\\
Advection of velocities: Characteristic\\
Advection of tracers: edge by edge explicit finite volume Leo Postma
scheme for tidal flats
%
\subsection{Definition of sources}
%
position source 1: $x \approx$ -21.6~m, $y \approx$ 5.3~m, $z$ = -0.5~m\\
position source 2: $x \approx$ -0.8~m, $y \approx$ -10.~m, $z$ = -0.5~m\\
constant discharge of 1.0~m3/s at both sources\\
tracer concentration at sources: 10.~g/L (or kg/m$^3$)\\
source 1 discharges tracer 1 and 2\\
source 2 discharges tracer 2 and 3\\
source 2 has an initial velocity: $U$ = 0.5~m/s $V$ = 2.0~m/s
%
\subsection{Comments}
%
% - Results:
%     We comment in this part the numerical results against the reference ones,
%     giving understanding keys and making assumptions when necessary.
%
%
\section{Results}
%
Figure 3.7.1 shows the horizontal mesh and sources positions in the
upper panel.
The lower panel highlights the influence of source 2 initial velocity
on the horizontal velocity field at mid depth at 550~s.
Additionally, the lower panel shows tracer 2 spreads in every direction
at source 1, unlike at source 2 where tracer 2 diffuses in the initial
velocity direction.
The horizontal and vertical plumes of each tracer at 1,100~s, on
figures 3.7.2 and 3.7.3 respectively, allows verifying that tracer 2
plume is the combination of tracer 1 and 3 plumes.
Moreover, the following mass balance of the \telemac{3d} simulation
shows that the amount of water injected by the sources is correct
(2,200~m$^3$ = 2 sources $\times$ 1~m$^3$/s $\times$ 1,100~s
with an error of 6 $\times$ 10$^{-4}$).
The mass balance also shows the conservation and the amount of
discharged tracer 1, 2 and 3 is correct.
(e.g. for tracer 2: 10~kg/m$^3 \times$ 2 sources $\times$ 1,100~s
$\times$ 1~m$^3$/s = 22,000~kg with an error of 5. 10$^{-2}$).

\begin{lstlisting}[language=TelFortran]
        BILAN DE MASSE FINAL      T=     1100.0000
        --- EAU ---
        MASSE INITIALE (DEBUT DE CE CALCUL) :            4000.000
        MASSE FINALE                              :      6200.001
        MASSE SORTIE DU DOMAINE(OU SOURCE) :            -2200.000
        MASSE PERDUE                              :     -0.6358517E-03
        --- TRACEUR 1 ---
        MASSE INITIALE (DEBUT DE CE CALCUL)        :     0.000000
        MASSE FINALE                               :      10999.98
        MASSE SORTIE (FRONTIERES OU SOURCE) :            -11000.00
        MASSE PERDUE                                :    0.1617723E-01
        --- TRACEUR 2 ---
        MASSE INITIALE (DEBUT DE CE CALCUL)          :     0.000000
        MASSE FINALE                                 :     21999.95
        MASSE SORTIE (FRONTIERES OU SOURCE) :            -22000.00
        MASSE PERDUE                                :     0.4578111E-01
        --- TRACEUR 3 ---
        MASSE INITIALE (DEBUT DE CE CALCUL)          :     0.000000
        MASSE FINALE                                 :     10999.97
        MASSE SORTIE (FRONTIERES OU SOURCE) :             -11000.00
        MASSE PERDUE                                  :   0.3159172E-01
\end{lstlisting}
%
\section{Conclusion}
%
\telemac{3d} is able to compute the evolution and conservation of a
tracer discharged by a source.
%
% Here is an example of how to include the graph generated by validateTELEMAC.py
% They should be in test_case/img
%\begin{figure} [!h]
%\centering
%\includegraphics[scale=0.3]{../img/mygraph.png}
% \caption{mycaption}\label{mylabel}
%\end{figure}
%
% bibliography
%\section{Reference}
